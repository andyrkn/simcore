\chapter{Algoritmul Genetic}

Problema propusa algoritmului genetic este gasirea unei traiectorii catre o tinta intr-un spatiu 3D cu posibile obstacole dat un proiectil fix si o forta de aplicare $F = 50N$ cu $dt=50ms$. 

S-a inceput cu o populatie $\lambda = 100$ si 50 de generatii pentru teste relativ simple.

\section{Cromozom}
Vectorul forta $F$ necesar pentru a reprezenta o traiectorie este $F(F_{x},F_{y},F_{z})$, fiecare reprezentand forta aplicata pe fiecare axa exprimata in newtoni. De asemenea in cazul reflectarii de pe suprafete/alte obstacole fiecare obiect are nevoie de un anumit factor de elasticitate $\mu \in [0,0,.7]$. 

In acest caz putem forma un cromozom din 4 numere naturale, $F$ putand fi exprimat in urmatorul mod:
\begin{itemize}
    \item $F_{x}$, 6 biti ce reprezinta o forta $F \in [0,63]$
    \item $F_{y}$, 6 biti ce reprezinta o forta $F \in [0,63]$
    \item $F_{z}$, 6 biti ce reprezinta o forta $F \in [0,63]$
    \item $\mu$, 3 biti ce reprezinta o elasticitate $\mu \in [0,7]$
\end{itemize}

Astfel un cromozom este un sir de lungime fixa pentru orice individ de lungime $l = 21$.

\begin{center}
    $\underbrace{001010}_{10}$ $\underbrace{001111}_{15}$ $\underbrace{000000}_{0}$ $\underbrace{100}_{4}$ \linebreak \linebreak
    \textit{Exemplu un cromozom cu $F(10,15,0), \mu=0.4$}
\end{center}

Lungimea de 6 biti pentru $F_{x},F_{y},F_{z}$ a fost aleasa cu motivatia ca o forta de $63N$ este de ajuns pentru a muta un cub cu masa de $1kg$ intr-un spatiu relativ mic, ceea ce este de ajuns pentru a crea mai multe nivele de dificultate. 

Pentru populatia initiala fiecare cromozom primeste 4 valori randomizate pentru cele 4 variabilare posibile. Alte posibilitati au fost randomizarea unei singure variabile sau a unei valori de 21 de biti. Dupa fiecare calcul al fitnessului, indivizii sunt sortati dupa fitness pentru a fi pregatiti pentru selectie. 
\section{Fitness}

Fitnessul se calculeaza in doua moduri in functie de rezultatul fiecarui individ:
\begin{itemize}
    \item In cazul in care un individ loveste tinta, atunci $f(x) = dt$, unde- $dt$ este timpul in care individul ajunge din punctul de start la tinta.
    \item In caz contrar, $f(x) = d(x) * max_{dt}$, unde $d(x)$ este distanta individului fata de tinta in momentul $max_{dt}$, iar $max_{dt}$ este durata de timp maxima alocata unui test.
\end{itemize}  

Pentru rularea cat mai rapida a testului, coliziunea dintre indivizi este dezactivata pentru a putea fi testata o intreaga generatie in acelasi timp. Acest lucru permite si vizualizarea unei generatii.

\section{Selectie}
 
Modul de selectia utilizat a fost Tournament Selection.

In fiecare generatie pot exista un total de $n_{t}$ tournamnets, $n_{t} \in [1,3]$. Pentru fiecare tournamnet se alege in mod uniform din populatie un individ ce nu a castigat deja un tournament anterior. In fiecare tournament pot exista maxim $3*n_{t}$ participanti, iar cei mai buni 2 indivizi din fiecare tournament sunt selectati pentru crossover,mutatie si sunt adaugati in generatia urmatoare. Pentru a crea noua populatie, sunt adaugati in ordinea fitnessului cromozomii din generatia precedenta pana cand $\lambda = 100$.

\section{Crossover}
Crossover se realizeaza in mod classic intre doi parinti, dupa un indice $i_{1}$ complet random. Prima parte a primului parinte va fi legata de a doua parte a partenerului.

Exemplu crossover intre un cromozom cu $\{F_{1}(10,15,0), \mu_{1} = 4\}$ si $\{F_{2}(33,11,5),$ \linebreak $\mu_{7} = 4\}$ la indicele $i=11$:
\begin{center}
    $\underbrace{001010}_{10}$ $\underbrace{001111}_{15}$ $\underbrace{000000}_{0}$ $\underbrace{100}_{4}$ \linebreak \linebreak
    $+$ \linebreak
    $\underbrace{100001}_{33}$ $\underbrace{001011}_{11}$ $\underbrace{000101}_{5}$ $\underbrace{111}_{7}$ \linebreak \linebreak
    $\Downarrow$ \linebreak
    $\underbrace{001010}_{10}$ $\underbrace{001111}_{15}$ $\underbrace{000101}_{5}$ $\underbrace{111}_{7}$ \linebreak \linebreak
\end{center}

In TwoPointCrossOver, se genereaza un al doilea indice $i_{2}$ care indica unde va fi introdus restul genomului primului parinte, genomul partenerului fiind plasat intre $i_{1}$ si $i_{2}$.

Numarul de indivizi prezenti in populatia $i+1$ rezultati prin crossover este maxim $n_{t}$ unde $n_{t}$ este numarul de tournamente ce au avut loc in populatia $i$.
\section{Mutatie}

Folosind tournament selection, mutatiile au loc doar asupra castigatorilor. Fiecare castigator sufera o mutatie de un bit asupra cromozomului, iar cromozomul mutat ajunge in generatia urmatoare. Alte modalitati sunt: 

\begin{itemize}
    \item Posibila mutatie a oricarui cromozom din populatie fata doar de campioni.
    \item Procentaj de mutatie fata de o mutatie fixa de 1 bit.
\end{itemize}

Ambele modalitati sunt viabile si au fost abordate in Rezultate.
