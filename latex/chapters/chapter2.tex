\chapter{Tehnologii Folosite}

intro

\section{C++}
 
Mediul de lucru este C++17 \footnote{\url{http://www.cplusplus.com/}}sub Visual Studio 2019. C++ a fost ales in principal datorita performantei si a experientei anterioare. Prin intermediul OOP au fost create structuri
pentru o implementare cat mai naturala a motorului fizic. 

\section{OpenGL}
 
Reprezentarea grafica se realizeaza cu ajutorul librariei OpenGL 3.3\footnote{\url{https://www.opengl.org/}}. Motivatia librariei OpenGL a fost atat introducerea ei in Anul 3 cat si permitarea implementarii cat mai primitiva a
motorului fizic.

\section{OpenGL Mathematics (GLM)}
 
Fiind un motor fizic, este necesar ca toate calculele matematice sa fie realizate cat mai corect si rapid, astfel libraria GLM \footnote{\url{https://glm.g-truc.net/0.9.9/index.html/}} a fost aleasa pentru viteza si compatibilitatea ei cu OpenGL.
Principalele structuri folosite din GLM au fost:
\begin{itemize}
    \item glm::vec2/glm::vec3/glm::vec4 reprezentand vectori. 
    \item glm::quat, (glm quaternions implementati sub forma w,x,y,z). Folositi in principal pentru reprezentarea orientarii obiectelor. Motivatia quaternionilor in favoarea 
    matricelor de rotatie a fost performanta si evitarea problemelor unghiurilor euler, cum ar fi Gimbal Lock \footnote{\url{https://en.wikipedia.org/wiki/Gimbal_lock/}}
    \item glm::mat3/glm::mat4, Matrici patratice, in principal pentru matricile MVP.
\end{itemize}
Pe langa structuri, metodele esentiale:
\begin{itemize}
    \item glm::rotate, folosit pentru a roti quaternioni dupa o axa de rotatie cu un anumit unghi
    \item glm::translate, pentru translatea matricii de translatie
    \item glm::dot/glm::cos
    \item glm::mat4\_cast pentru conversia din quaternioni in matrici de rotatie.
\end{itemize}
Alternative au fost: Eigen \footnote{\url{http://eigen.tuxfamily.org/index.php?title=Main_Page}} si CML \footnote{\url{https://github.com/demianmnave/CML/wiki/The-Configurable-Math-Library}}.Toate librariile fiind testate in \url{https://github.com/mfoo/Math-Library-Test}.

\section{GLFW}
 
Pentru initializarea ferestrei OpenGL a fost folosit \footnote{\url{https://www.glfw.org/}} GLFW. Aceasta librarie ofera in mod simplistic un context OpenGL si mai multe optiuni cum ar fi VSync sau DoubleBuffering. De asemenea,
keyboard si mouse events sunt capturate prin GLFW.

Alternative au fost : GLUT \footnote{\url{http://freeglut.sourceforge.net/}}. GLFW a fost ales in favoarea GLUT deoarece pe masina unde a fost dezvoltata aplicatia, GLUT nu reusea sa intanstieze OpenGL 3.3+.

\section{Concluzii}

C++ impreuna cu OpenGL si GLM ofera un mediu de lucru cat mai aproape de procesor si in acelasi timp un set de structuri si unelte pentru dezvoltarea naturala a Motorului Fizic.