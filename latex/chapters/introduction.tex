\chapter*{Introducere} 
\addcontentsline{toc}{chapter}{Introducere}

Algoritmii genetici sunt un tip al algoritmilor de optimizare. Acestia cauta solutii optime intr-un spatiu larg,unde algoritmii generici ar fi prea greoi, sau unde nu exista un algoritm determinist. Acestia cauta sa minimizeze sau sa maximezeze o anumita functie data. Algoritmii genetici folosesc principiile selectiei si evolutiei, observate in natura pentru a produce mai multe solutii unei singure probleme. Datele de intrare ale unui algoritm genetic sunt un set de posibile solutii, iar apoi algoritmul genetic incearca sa imbunateasca solutiile pe parcursul a sute sau mii de generatii. Prin evolutie se presupune ca fiecare generatie este mai buna decat precedenta, astfel dupa multe iteratii algoritmii genetic pot produce solutii noi, eficiente care in mod normal nu ar putea fi gasite de mintea umana. Un astfel de exemplu sunt antele create de un algoritm genetic, (detaliat in Capitolul 1.2.1 Evolved Antenna).

Acest tip de algoritmi au aparut prima data in 1950 cand Alan Turing a propus o "Masina care invata" ce ar imita principiile evolutiei. Mai tarziu in anii 1989 a aparut primul algoritm genetic comercializat. Acesta rezolva probleme de optimizare date de utilizator.

Intelegerea si replicarea evenimentelor fizice este un subiect in continua cercetare in ziua de azi. Pentru a intelege cat mai bine natura, algoritmii evolutivi au inceput sa creasca nivelul de abstractizare cat mai mult pentru a obtine rezultate cat mai fidele cu ceea ce vedem in natura. De exemplu pentru a simula locomotia, optimizarea modului in care este controlata o incheietura (1.1.3 Joint Evolution) aduce rezultate modeste, rigide care nu par naturale. Asa ca dupa incheieturi a fost abstractizat sistemul muscular (1.1.2 Muskoscheletal Evolution) iar apoi pana si celulele ce creeaza un corp (1.1.1 Cell Evolution) pentru a obtine cele mai fidele rezultate prin evolutie. In 2017 Google Deep Mind AI \footnote{\url{https://www.youtube.com/watch?v=gn4nRCC9TwQ}} au reusit sa creeze un individ care alearga pe cont propriu, avand un corp similar corpului uman.

Aceasta lucrare isi propune sa rezolve o problema de optimizare folosind algoritmi genetici pentru a nimeri o tinta intr-un mediu fizic similar cu cel terestru. Adaugand obstacole atat statice cat si dinamice, problema de optimizare poate deveni o problema cu optim dinamic, in care pot exista mai multe solutii eficiente. Se cauta control total atat asupra evolutiei indivizilor cat si a mediuliu fizic in care acestia cauta solutia, astfel nu se va folosi nici un fel de librarie deja existenta pentru suport fizic sau algoritmi genetici. 

Algoritmul genetic incearca sa rezolve problema prin a aplica un vector forta unui corp ce are o anumita elasticitate, iar apoi acesta imbunatateste solutia evoluand vectorul forta si elasticitatea obiectului. Incepem intai prin a defini mediul fizic (3. Mediul Fizic) apoi algoritmul genetic (4.Algoritmul Genetic) si rezultate acestuia in urma experimentelor (5. Rezultate).