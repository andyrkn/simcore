\chapter*{Contributii} 
\addcontentsline{toc}{chapter}{Motivație}

Pentru a atinge scopul propus, a fost intai nevoie de dezvoltarea unui motor fizic ce poate aproxima toate actiunile fizice care pot aparea in cadrul unei simulari. 

Incat acesta este responsabil de functia fitness, motorul fizic trebuie sa poate rula atat intr-un mod in care putem vizualiza o intreaga generatie, sau un singur individ (head), cat si intr-un mod in care acesta returneaza fitnessul cat mai rapid posibil algoritmului genetic, vizualizarea generatiilor nefiind necesara in pasul de evolutie. 

In cadrul algoritmului genetic a fost cautat cel mai eficient mod de a crea un cromozom si diferite moduri in care acesta sa evolueze astfel incat sa rezolve problema indifierent de mediul in care acesta se gaseste. A fost urmarita schimbarea in generatii in functie de selectia folosita, recombinarea cat si mutatia.